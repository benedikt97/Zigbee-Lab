\chapter{Life Cycle Management}
\section{Deployment}
In diesem Kapitel geht es um die Pflege, die Bereitstellung sowie das Zurücksetzen des Praktikumversuchs. Sämtliche Schritte wurden mit Ansible-Playbooks
automatisiert.

Folgende Schritte müssen ausgeführt werden, um eine Raspberry vorzubereiten.
\begin{itemize}
    \item Installation von Raspbian OS
    \item Anlegen User ansible / ansible
    \item Kopieren von Ansible-Playbooks auf Raspberry (bspw. SCP)
\end{itemize}

Zum Ausrollen werden die Rollen \grqq Docker\grqq{} und \grqq ZigbeeLab\grqq{} zugewießen.
\begin{lstlisting}
    ansible-playbook DeployLab.yaml -K -e "channel=<channelnumber>"
\end{lstlisting}

Anschließend wird
\begin{itemize}
    \item Docker Installiert
    \item Der User Student angelegt
    \item Die /etc/hosts Einträge gesetzt
    \item Die Docker Umgebung vorbereitet im Ordner srv
    \item Die Docker Services gestartet (zigbee2mqtt)
\end{itemize}

\section{Zurücksetzen des Versuchs}

Zum zurücksetzen wird das Playbook \grqq ResetLab\grqq{} ausgeführt.

\begin{lstlisting}
    ansible-playbook ResetLab.yaml -K -e "channel=<channelnumber>"
\end{lstlisting}

\section{Update der eingesetzten Software}

Die RaspberryOS Pakete können mit dem Paktetmanager apt aktuell gehalten werden. Docker zieht initial die Container aus Dockerhub.
Mit folgendem Befehl überprüft Docker das Repository auf aktuelle Container.

\begin{lstlisting}
    docker compose pull /srv/docker-compose.yaml
\end{lstlisting}

Anschließend können die Container mit folgendem Befehl neu gestartet werden.

\begin{lstlisting}
    docker compose up -d /srv/docker-compose.yaml
\end{lstlisting}

\section{Troubleshooting}

Es empfiehlt sich bei Problemen die Container zu stoppen und erneut ohne den Parameter -d zu starten.

\begin{lstlisting}
    docker compose up /srv/docker-compose.yaml
\end{lstlisting}

Nun kann der vollständige Log beobachtet werden, welcher zumeißt hilfreich ist.
Alternativ kann man sich den Log eines Containers mit folgendem Befehl Live betrachten:

\begin{lstlisting}
    docker attach <container-name>
\end{lstlisting}







