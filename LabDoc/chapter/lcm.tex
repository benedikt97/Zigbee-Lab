\chapter{Life Cycle Management}

In diesem Kapitel geht es um die Pflege, die Bereitstellung sowie die Zuücksetzung des Praktikumversuchs. Sämtliche Schritte wurden mit Ansible-Playbooks
automatisiert.

Um ein Raspberry automatisch für einen Versuch vorzubereiten, muss dessen Ip-Adresse in der \grqq hosts \grqq{} Datei eingetragen werden. Anschließend muss der 
SSH Schlüssel des Ansible Servers auf dem Raspberry hinterlegt werden. Auf dem Raspberry muss ein User \grqq ansible \grqq{} angelegt.

Folgende Schritte müssen ausgeführt werden, um eine Raspberry vorzubereiten.
\begin{itemize}
    \item Installation von Raspbian OS
    \item Anlegen User ansible / <secret>
    \item Hinterlegen SSH-Key von Ansible Server für ansibe user
    \item Eintragen Raspberry IP in hosts in Ansible Root Verzeichnis
    \item Konnektivität Ansibe Server und Raspberry herstellen (ping)
\end{itemize}

\subsection{Deployment}

Zum ausrollen folgenden Befehl auf dem Ansible Host absetzen.

Anschließend wird
\begin{itemize}
    \item Docker Installiert
    \item Der User Student angelegt
    \item Die config Files ausgerollt
    \item Die /etc/hosts Einträge gesetzt
    \item zbwireshark Installiert
    \item Die Docker Services gestartet (zigbee2mqtt)
\end{itemize}

\begin{lstlisting}
    ansible-playbook DeployLab.yaml -K
    
    #BECOME Password:
    <secret>
\end{lstlisting}

\subsection{Zurücksetzen des Versuchs}

\subsection{Update der eingesetzten Software}




