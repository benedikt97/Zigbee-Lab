\chapter{Life Cycle Management}
\section{Deployment}
In diesem Kapitel geht es um die Pflege, die Bereitstellung sowie das Zurücksetzen des Praktikumversuchs. Sämtliche Schritte wurden mit Ansible-Playbooks
automatisiert.

Folgende Schritte müssen ausgeführt werden, um ein RaspberryPi vorzubereiten.
\begin{itemize}
    \item Installation von Raspbian OS 32 Bit
    \item Anlegen eines Users
    \item Klonen des GitHub Repositorys 
    \item Installation von Ansible
    \item UART auf Raspberry vorbereiten
    \item Ansible Playbook Ausführen
\end{itemize}

Das Repository kann mit Git und folgendem befehl geklont werden:
\begin{lstlisting}
    git clone https://github.com/nlab4hsrm/ZigbeeHSRM.git
\end{lstlisting}
Am besten wird das Repository direkt in den Homefolder des Users geklont, der später den Versuch durchführt. Damit kann dieser selbständig den Versuch zurücksetzen. Im Anschluss liegt ein Ordner
\grqq ZigbeeHSRM\grqq{} ab. In diesem sind die Playbooks und der LabGuide enthalten.\\
Zusätzlich muss noch Ansible installiert werden:
\begin{lstlisting}
    sudo apt install ansible
\end{lstlisting}

Bevor der Versuch ausgerollt werden kann, ist es zusätzlich notwendig manuell die UART Schnittstelle des Raspberrys zu konfigurieren. Dazu wird zuerst das Tool rasp-config aufgerufen.
\begin{lstlisting}
    sudo raspi-config
\end{lstlisting}
Hier muss unter dem Menüpunkt \grqq 3 Interface Options\grqq{} die Konfiguration für \grqq I6 Serial Port\grqq{} gestartet werden. Im ersten Schritt wir der Zugriff auf die Shell über den seriellen
Port deaktiviert. In zweitem Schritt wird die Serielle Schnittstelle dann aktiviert. 

Anschließend wird Bluetooth deaktiviert. Dies passiert durch hinzufügen folgender Zeile in die Datei \grqq /boot/config.txt\grqq{}
\begin{lstlisting}
    dtoverlay=disable-bt
\end{lstlisting}

Im letzten Schritt muss noch der Dienst \grqq hciuart\grqq{} deaktiviert werden, da dieser ansonsten auf die Schnittstelle zugreift
\begin{lstlisting}
    systemctl disable hciuart
\end{lstlisting}

Danach muss der RaspberryPi neu gestartet werden! 


Zum Ausrollen werden die Rollen \grqq Docker\grqq{} und \grqq ZigbeeLab\grqq{} zugewießen. Dies geschieht durch die DeployLabl.yaml.
\begin{lstlisting}
    ansible-playbook ~/ZigbeeHSRM/playbooks/DeployLab.yaml -K -e "channel=<channelnumber>"
\end{lstlisting}
Im Anschluss muss das jeweilige Passwort des Users angegeben werden. Der User benötigt Root Rechte.

Anschließend wird
\begin{itemize}
    \item Docker Installiert.
    \item Die /etc/hosts Einträge gesetzt.
    \item Die Docker Umgebung vorbereitet.
    \item Die Versuchsumgebung mit allen Komponenten gestartet.
\end{itemize}

\section{Zurücksetzen des Versuchs}

Zum zurücksetzen wird das Playbook \grqq ResetLab\grqq{} ausgeführt.

\begin{lstlisting}
    ansible-playbook ~/ZigbeeHSRM/playbooks/ResetLab.yaml -K -e "channel=<channelnumber>"
\end{lstlisting}
Hinweis: Das Playbook funktioniert nur in der 32-Bit Version von RaspbianOS. Für 64-Bit muss ein anderes Docker Repository im Playbook angegeben werden.
Es wird dann arm64 anstelle von armhf benötigt. Dies ist allerdings nicht getestet.

\section{Update der eingesetzten Software}

Die RaspberryOS Pakete können mit dem Paktetmanager apt aktuell gehalten werden. Docker zieht initial die Container aus Dockerhub.
Mit folgendem Befehl überprüft Docker das Repository auf aktuelle Container.

\begin{lstlisting}
    docker compose pull /srv/docker-compose.yaml
\end{lstlisting}

Anschließend können die Container mit folgendem Befehl neu gestartet werden.

\begin{lstlisting}
    docker compose up -d /srv/docker-compose.yaml
\end{lstlisting}

Sämtliche Software wird onehin bei jedem Zurücksetzen des Versuchen aktualisiert.

\section{Troubleshooting}

Es empfiehlt sich bei Problemen sich den Log der Container anzuschauen.

\begin{lstlisting}
    docker attach <container-name>
\end{lstlisting}

In der Regel ist es am sinnvollsten, mit dem Log von zigbee2mqtt zu starten. Dieser ist erfahrungsgemäßg bei einer Fehlersuche sehr hilfreich.

Ein gängier Fehler ist, das der Container zigbee2mqtt regelmäßig neustartet. Diese Information erhällt man mit einem
\begin{lstlisting}
    docker ps
\end{lstlisting}
Die Gründe dafür können vielfältig sein. Am häufigsten ist ein Problem mit dem Zugriff auf die UART Schnittstelle. Ebenfalls kann ein belegter Funkkanal
oder eine doppelte PAN-ID in der Umgebung Grund für Probleme sein. 
Der Grund kann aus der Log des Containers ermittelt werden:
\begin{lstlisting}
    docker attach zigbee2mqtt
\end{lstlisting}



