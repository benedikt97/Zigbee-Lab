\chapter{Life Cycle Management}
\section{Deployment}
In diesem Kapitel geht es um die Pflege, die Bereitstellung sowie das Zurücksetzen des Praktikumversuchs. Sämtliche Schritte wurden mit Ansible-Playbooks
automatisiert.

Folgende Schritte müssen ausgeführt werden, um ein RaspberryPi vorzubereiten.
\begin{itemize}
    \item Installation von Raspbian OS
    \item Anlegen eines Users
    \item Klonen des GitHub Repositorys nlab4hsrm zigbeelab
\end{itemize}

Zum Ausrollen werden die Rollen \grqq Docker\grqq{} und \grqq ZigbeeLab\grqq{} zugewießen. Dies geschieht durch die DeployLabl.yaml.
\begin{lstlisting}
    ansible-playbook DeployLab.yaml -K -e "channel=<channelnumber>"
\end{lstlisting}

Anschließend wird
\begin{itemize}
    \item Docker Installiert.
    \item Die /etc/hosts Einträge gesetzt.
    \item Die Docker Umgebung vorbereitet.
    \item Die Versuchsumgebung mit allen Komponenten gestartet.
\end{itemize}

\section{Zurücksetzen des Versuchs}

Zum zurücksetzen wird das Playbook \grqq ResetLab\grqq{} ausgeführt.

\begin{lstlisting}
    ansible-playbook ResetLab.yaml -K -e "channel=<channelnumber>"
\end{lstlisting}

\section{Update der eingesetzten Software}

Die RaspberryOS Pakete können mit dem Paktetmanager apt aktuell gehalten werden. Docker zieht initial die Container aus Dockerhub.
Mit folgendem Befehl überprüft Docker das Repository auf aktuelle Container.

\begin{lstlisting}
    docker compose pull /srv/docker-compose.yaml
\end{lstlisting}

Anschließend können die Container mit folgendem Befehl neu gestartet werden.

\begin{lstlisting}
    docker compose up -d /srv/docker-compose.yaml
\end{lstlisting}

\section{Troubleshooting}

Es empfiehlt sich bei Problemen sich den Log der Container anzuschauen.

\begin{lstlisting}
    docker attach <container-name>
\end{lstlisting}

In der Regel ist es am sinnvollsten, mit dem Log von zigbee2mqtt zu starten. Dieser ist erfahrungsgemäßg bei einer Fehlersuche sehr hilfreich.







