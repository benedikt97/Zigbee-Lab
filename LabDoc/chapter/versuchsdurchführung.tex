\chapter{Versuchsdurchführung}

\section{Ursprungszustand vor Versuch}

Der ZigBee Kordinator ist als \grqq Hat \grqq{} auf dem Raspberry installiert. Der Sniffer ist per USB an der 
Frontseite des Raspberrys angeracht. Eingabegeräte, ein Monitor sowie die Stromversorgung sind weiterhin vom
Studenten anzuschließen.  

Auf einem RaspberryPi werden die Anwendungen zigbee2mqtt, Mosquitto sowie HomeAssistant der Docker ausgeführt. Die
Services sind konfiguriert, es sind keine Geräte per Zigbee verbunden. 
Die jeweligen Webinterfaces sind über eine Webadresse im Browser erreichbar. Der ZigBee Kordinator ist als \grqq Hat \grqq{}
auf dem Raspberry installiert. Der Sniffer ist per USB an der Frontseite des Raspberrys angeracht.  

\section{Aufgabenstellungen}

\subsection{Aufgabe 1 - Vertraut machen mit der Umgebung}

Der Student soll den Webbrowser starten, und z2m.local aufrufen. Er soll ich Anhand einer kurzen Beschreibung selbst
mit der WebGui vertraut machen. Der Koordinator soll auf den entsprechenden Kanal eingestellt werden.

Der Student soll sichergehen, dass keine Geräte in zigbee2mqtt registriert sind. Falls dies der Fall ist, soll
er das Script zum zurücksetzen des Versuches starten.

Der Student soll Wireshark über die Kommandozeile mit den entsprechendem Befehl starten, und überprüfen ob die Anwendung
ohne Fehlermeldung startet und orgnungsgemäß funktioniert.

\subsection{Aufgabe 2 - Joining der Phillips Hue Lampe und Fernbediernung}

Der Student soll das Beitreten von Komponenten in zigbee2mqtt erlauben. Erst anschließend soll er Wireshark starten.
Nun lässt er die Phillips Hue Lampe sowie die Fernbedienung dem Netzwerk beitreten. Sobald zigbee2Mqtt ein erfolgreiches
Interview meldet, wird Wireshark gestoppt, und der Sniffing Vorgange abgespeichert.




