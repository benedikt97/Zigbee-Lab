\chapter{Versuchsdurchführung}

\subsection{Aufgabe 2 - Joining einer Phillips Hue Lampe}
\subsection{Aufgabe 3 - Joining einer Fernbedienung über die Lampe}
\subsection{Aufgabe 4 - Binding der Fernbedienung}
\subsection{Aufgabe 5 - Gruppenbildung}
\subsection{Fragen}
\subsubsection{Aufgabe 2}

\begin{Fragen}
    1. Untersuchen sie den \textbf{Beacon-Request}\\
    Erläutern sie den Frametype und den Command Identifier.\\ 
    Erläutern die Ziel- und Quelladdresse und was sie daraus erschließen können.\\
    
    2. Wie teilt der Koordinator den umliegenden Geräten ab, dass er dem Netzwerk den Beitritt weiterer Geräte erlaubt ?\\
    Zu welchem Frametype gehört der Beacon und durch welchen Wert wird er spezifiziert?\\
    Welche Ziel- und Quelladressen werden verwendet?\\
    
    4. Welchen Wert hat das Feld „Association Permit“ im letzten Beacon des Koordinators und wie ist dieser Wert zu interpretieren? \\
    
    5. Untersuchen Sie den \textbf{Association Request} der Lampe an den Koordinator. \\
    Welchen Wert hat das Feld \grqq Allocate Address \grqq{} und wie ist dieser zu interpretieren?\\
    Welchen Wert hat das Feld \grqq Device Type\grqq{} und wie ist dieser zu interpretieren?\\
    
    7. Untersuchen die den "Data Request" von der Lampe an den Koordinator.\\
    Erläutern Sie die Funktion dieser Nachricht.\\
    Mit welchem Kommandoframe antwortet der Koordinator auf den Data-Request?\\
    Welche Werte besitzen die Felder „Short Address“ und „Association Status“?\\
    
    8. Untersuchen Sie einen \textbf{IEEE802.15.4. Ack-Frame}.\\
    Welche Adressfelder werden benutzt? \\
    Wie findet die Zuordnung zum Daten- oder Kommandoframe statt, der durch die Ack bestätigt wird? \\
    Werden grundsätzlich alle Frames bestätigt?\\
    
    10. Wie lautet die letzte Nachricht, bei der 64-bit-MAC-Adressen verwendet werden und wie\\
     lautet die erste Nachricht, bei der die 16-Bit-Kurzadresse der beigetretenen Lampe verwendet wird?\\
    
    11. Was ist die letzte Nachricht, die auf dem NWL-Layer unverschlüsselt übertragen wird?\\
    
    12. Erläutern Sie den Zweck der \textbf{Tranport-Key-Nachricht}.\\
     Wie lautet der Frametype des 802.15.4-Frames, in dem die Transport-Key-Nachricht transportiert wird? \\
     Wie lautet der ZigBee-NWK Frametype des Frames?\\
     Treffen Sie möglichst genaue Aussagen zum in der Transport-Key übertragenen Schlüssel (Schlüsseltype). \\
     Erläutern Sie, wie die Transport-Key-Nachricht kryptographisch gesichert ist.\\
     Interpretieren Sie den Inhalt des Radius Feldes im NWK-Frame, das die TransportKeyNachricht enthält!\\
    
    13. Erläutern Sie, den Zweck des versendeten \textbf{Active-Endpoint-Requests} und des \textbf{SimpleDescriptor-Requests}.
     Beschreiben Sie die Information, die in den entsprechenden Response-Nachrichten enthalten ist. 
     Wie stellt deConz die Information dar?
     Welche Endpoints werden für den Austausch der untersuchten Request- und ResponseNachrichten verwendet? Interpretieren Sie dies!

    
    14. Durch welche ZigBee-Frames werden die Schaltvorgänge übertragen? \\
    
    15. Beschreiben Sie möglichst genau, durch welche Headerfelder die Schaltvorgänge definiert sind! \\
        
    16. Welche Endpoints werden für die Schaltvorgänge benutzt? Woher hat der Koordinator Kenntnis über die in der Lampe verwendeten Endpoints? \\
    \end{Fragen}
    \subsubsection{Aufgabe 3}
    \begin{Fragen}
        1. Erläutern Sie den Zweck der \textbf{Permit-Join-Request} Nachricht. An welche ZigBee-NWKZieladresse wird die Nachricht versendet? 
        Erläutern Sie das wichtigste Headerfeld!\\
        
        2. Welchem Zweck dient die \textbf{Update Device} Nachricht? Wer ist Absender und wer ist
        Empfänger? Welche Adresse steht im Feld „Device Address“?\\
        
        3. Von welchem Device erhält die Fernbedienung ihre 16 Bit Kurzadresse und wie lautet sie?\\
        
        4. Wie viele \textbf{Transport-Key} Nachrichten wurden ausgetauscht? Erläutern Sie wer jeweils
        der Absender und wer der Empfänger ist. Versuchen Sie die den Vorgang zu erklären und
        gehen Sie dabei auf das Kommandoframe \grqq Tunnel\grqq{} ein. Wie sind die Transport-Key Nachrichten kryptographisch gesichert?
        Was sind die wichtigsten Headerfelder des Tunnel-Kommandoframes?\\
        
        5. Untersuchen Sie die \textbf{Device Announcement} Nachricht der Fernbedienung, welchen
        Zweck hat sie? Schauen Sie sich die \grqq Capability Information \grqq{} an. Handelt es sich um ein
        Full-Function-Device? Welcher Wert steht im Feld „AC Power“ und was sagt dieser Wert aus? \\
        
        6. Untersuchen Sie die \textbf{Active-Endpoint-Request} Nachricht und ihren Weg vom
        Koordinator bis zur Fernbedienung. Vergleichen Sie die Adressen im ZigBee-NWKLayer und im IEEE-Layer und erklären Sie den Zusammenhang. An welchem Headerfeld
        können Sie zweifelsfrei identifizieren, dass es die gleiche Nachricht ist, die nur weitergeleitet wird? \\
        
        7. Untersuchen Sie die \textbf{Simple Descriptor Response}- Nachrichten der Fernbedienung!
        Welche Informationen enthält diese Nachricht? \\
        \end{Fragen}
        \subsubsection{Aufgabe 4}
        \begin{Fragen}
            1. Untersuchen Sie die \textbf{Bind Request}- und die \textbf{Bind Response}-Nachricht. 
            Was sind jeweils die NWK-Quell- und NWK-Zieladressen? Erläutern Sie, den Inhalt der BindRequest Nachricht. 
            Was genau bewirkt die Nachricht? In der Nachricht sind nur 64-Bit Adressen enthalten. Stellt das ein Problem dar? \\
            
            2. Warum wird vom Koordinator kein Bind-Request an die Lampe gesendet. \\
            
            3. Betrachten Sie die \textbf{ZCL: OnOff} Nachricht. Geben Sie die NWK-Quell- und Zieladresse an. 
            Welchen Wert hat das On/OFF-Cluster und welches Kommando wird zum Schalten verwendet?\\
            
            4. Interpretieren Sie die von der Fernbedienung gesendeten \textbf{Data Request} Nachrichten?
            Wie groß ist der zeitliche Abstand zwischen zwei Data-Requests? Was löst eine DataRequest-Nachricht beim Empfänger aus? Geben Sie ein Beispiel. 
            \end{Fragen}

            \subsubsection{Aufgabe 5}

            \begin{Fragen}
                1. Verdeutlichen Sie sich den Vorgang in dem Sie die vom Koordinator versendeten
                Nachrichten \textbf{Get Group Membership} bzw. \textbf{Add Group} untersuchen. Fassen Sie die
                wichtigsten Informationen der Nachrichten zusammen. \\
                
                2. Betrachten Sie die \textbf{Add Group Response} Nachricht des Empfängers.
                Welche Information ist enthalten? Welcher Zielendpunkt wird im APS-Frame des AddGroup-Befehles verwendet? Geben
                Sie eine Erklärung! \\
                
                3. Wie lautet die Destination Adresse im ZDP-Header der \textbf{Bind Request} Nachricht?
                Welcher Zielendpunkt ist vorhanden? \\
                
                4. Was ist die NWK-Zieladresse der \textbf{ZCL-OnOff} Nachricht? Um welche Art von Nachricht handelt es sich hierbei?
                Finden Sie die von Ihnen eingestellte Gruppen-ID in der „ZCL OnOff“-Nachricht wieder? \\
                \end{Fragen}
                
1. Welchem Zweck dienen die „Link Status“ Nachrichten? Über welche Anzahl von „Hops“
wird diese Nachricht übertragen? Analysieren Sie exemplarisch einige Link-StatusNachrichten und Interpretieren Sie diese! \\


2. Untersuchen Sie die „Link Quality Request“- bzw. „Link Quality Response“-Nachrichten.
Gehen Sie auf den LQI-Wert in der „Link Quality Response“ Nachricht und was bedeutet
dieser? \\


3. Interpretieren Sie „Route Request“ Nachrichten und zugehörige „Route-Response“-
Nachrichten. 




