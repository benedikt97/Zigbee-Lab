\chapter{Grundlagen}

In diesem Kapitel werden alle verwendeten Komponenten und Technologien kurz erläutert.

\section{LR-WPAN - IEEE 802.15.4}

LR-WPAN \cite{lrwpan} steht für \grqq Low Rate - Wireless personal area network\grqq{}. Es handelt sich um ein drahtloses geschlossenes Netzwerk, welches für 
niedrige Datenraten ausgelegt ist. Der Standard definiert den Physical Layer sowie den Media-Access Layer und bildet die Grundlage von ZigBee, Thread und 6LowPAN.
Im Standard sind mehrere Modulationsverfahren sowie Frequenzbereiche definiert. Der Standard beinhaltet sowohl die klassische Stern Topologie, als auch die bei 
ZigBee verwendete Peer-to-Peer Topologie. Weiteres Schlüsselfeatures sind Kollisionsvermeidung sowie Echtzeitfunktionalitäten. Im Vergleich zum 802.1d Standard 
fallen vor allem die kürzeren Adressen auf. Dadurch kann die für den Anwendungsbereich wertvolle Bandbreite und Rechenleistung reduziert werden. 

\section{ZigBee}

Die ZigBee Alliance wurde durch ein Konsortium von Herstellern gegründet, um einen einheitlichen Übertragungsstandard
im Bereich Heimautomatisierung voranzubringen. ZigBee basiert auf dem offenem 802.15.4 Standard, bringt allerdings zusätzliche Komponenten mit.
ZigBee ist in Form von weiteren Protokollschichten implementiert, welche auf IEEE 802.15.4 aufsetzen. ZigBee nutzt DSSS, also Frequenzspreizung als Modulationsverfahren.
Die genutzten Kanäle, 11 bis 26, liegen im 2,4 Ghz Band. Zigbee interferiert damit mit WLAN und Bluetooth.

\begin{figure}[H]
  \centering
  \includegraphics[width=1\textwidth]{media/Zigbee Stack.jpg}
  \caption{ZigBee Protocoll Stack \\ Bildquelle: \url{https://www.researchgate.net/figure/IEEE820154-ZigBee-protocol-stack-architecture_fig2_265150617}}
\end{figure}

Der Anwendungsbereich für ZigBee ist die Heimautomatisierung. Geräte können zentral gesteuert und überwacht werden. 
Markante Eigenschaft von ZigBee ist, dass die Geräte keine direkte Funkverbindung
zu einem zentralen Controller brauchen. Andere Geräte können als Router fungieren, und damit die Reichweite erhöhen. Sende- und Empfangsleistung
ist vor allem bei kleinen batteriebetriebenen Geräten oft der einschränkende Faktor.


\section{Texas Instruments CC Chips}

Texas Instruments bietet ein Spektrum von Mikrocontrollern, die sich mit entsprechender Firmware für Devices als ZigBee Controller
nutzen lassen.Alternativ kann mit so einem Mikrocontroller auch ein Koordinator betrieben werden. Kleinere Varianten können in Endgeräten wie Lampen und Thermostate, 
größere als Koordinator selbst verwendet werden.

Die aktuelle Chipfamilie TexasInstruments CC26XX:

\begin{figure}[H]
  \centering
  \includegraphics[width=1\textwidth]{media/table26xx.png}
  \caption{TI Device Family}
\end{figure}

Als Koordinator werden die Leistungsfähigeren Chips aus der 265X Reihe eingesetzt. Touchlink ist eine Technologie, um ZigBee Geräte einfach zu koppeln. Dafür setzt
Touchlink Bluetooth LE ein. Die Unterstützung von diesem Protokoll ist daher vorteilhaft.

\begin{figure}[H]
  \centering
  \includegraphics[width=1\textwidth]{media/table265x.png}
  \caption{TI CC 265X Serie}
\end{figure}

In der Tabelle sind die unterstützten Protokolle der einzelnen Modelle sowie deren Leistungsfähigkeit aufgeführt.
Es ist anzumerken, dass die größeren Modelle schon den Standard Thread unterstützen, der vermutlich durch das
Projekt \grqq Matter\grqq{} erheblich an Bedeutung gewinnen wird.

Texas Instruments stellt als Basis für ZigBee Anwendungen eine Bibliothek \grqq Z-Stack\grqq{} zur Verfügung. Diese stellt grundlegenden 
Funktionen um das ZigBee Protokoll zu implementieren zur Verfügung. Mit Texas Instruments Code Composer Studio steht eine IDE bereit,
um den Entwicklungsprozess zu unterstützen. Auf den entsprechend Leistungsfähigeren Chips lassen sich in freie Speicherbereiche noch zusätzliche
Funktionalitäten einprogrammieren. Die Chips können mit Programmierboards des Herstellern programmiert werden. Alternativ kann man
günstig einen USB-Stick mit aufgelöteten CC Chip erwerben, und auch diesem mit entsprechenden Tools programmieren.

Weiter Informationen: \url{https://www.ti.com/tool/Z-STACK#overview}

In dem OpenSource Projekt Zigbee2mqtt werden ausschließlich Chips von Texas Instruments unterstützt. Die meisten gängigen Anbieter von Microchips 
haben entsprechende Modelle im Angebot. 

\section{Versuchshardware}

\subsection{RaspberryPi}

Der RaspberryPi ist ein ARM basierter Computer im Mini-Format. Er dient in diesem Versuch als Applikationsserver und gleichzeitig als Versuchs-PC, 
auf dem der Versuch durchgeführt wird. Die eingesetzten Anwendungen sind 
als Webservice implementiert und werden als Container mit Docker betrieben.

Der RaspberryPi besitzt die PC typischen Schnittstellen wie Ethernet, HDMI, und USB. Als Speicher wird eine SD-Karte eingesetzt. 

\subsection{RaspberryPi Zigbee Hat}

Als Zigbee Koordinator wird ein auf dem TI CC2652 basierendem RaspberryPi Hat vom Hersteller cod.m eingesetzt. Dieser wurde vom Hersteller
für den Einsatz mit \grqq homegear\grqq{} oder \grqq zigbee2Mqtt\grqq{} entwickelt. Ein Datenblatt und Bedienungsanleitung sind im Anhang.

\begin{figure}[H]
  \centering
  \includegraphics[width=1\textwidth]{media/codm.png}
  \caption{ZigBee cod.m Koordinator \\ Bildquelle: cod.m Produktfoto}
\end{figure}

Durch einen Lötjumper kann zwischen einer aufgedruckten und einer per \grqq ufl\grqq{} angeschlossenen Antenne gewechselt werden. Im Auslieferungszustand
ist es notwendig, eine externe Antenne anzuschließen.

\subsection{CC2531 Sniffer Stick}

Mit diesem Stick wird die ZigBee Kommunikation des Versuchsnetzwerkes mitgeschnitten.
Der Stick basiert auf einem leistungsschwachen Chip, der mit entsprechender Firmware ZigBee Kommunikation mitschneiden kann. 

Als Treiber wird ein in C geschriebenes Programm verwendet, welches es ermöglicht den Stick direkt als Interface in Wireshark hinzuzufügen. Der Quellcode findet sich in 
GitHub unter \url{https://github.com/andrebdo/wireshark-cc2531}. Eine Anleitung zum kompilieren ist im Repository enthalten. Die hieraus entstehende ausführbare Datei muss in ein
entsprechendes Verzeichnis kopiert werden, und kann anschließend als Interface in Wireshark ausgewählt werden. Die Funktion nennt sich bei Wireshark \grqq extcap\grqq{}.

\subsection{Phillips Hue Komponenten}

Unter dem Namen \grqq Hue\grqq{} vertreibt Phillips eine Reihe von vernetzten Endgeräten sowie entsprechende Komponenten um diese zu steuern.
Die Phillips Hue Serie setzt auf ZigBee sowie Bluetooth LE. Unter anderem sind Lampen, Steckdosen, eine Bridge sowie eine App verfügbar.
Die Bridge stellt bei traditionellen Lösungen die Schnittstelle zwischen dem ZigBee Netzwerk und der IP Kommunikation zu einer Smartphone
App oder ähnlichem. Die Endgeräte sind Kompatibel zu dem Software-Gateway Zigbee2mqtt, benutzen also keine speziellen Schlüssel.
Die Lampen werden in dem Versuch als Demonstrationsobjekte eingesetzt. Sie können Ein- und Ausgeschaltet werden, sowie gedimmt werden. Zusätzlich wird eine
Phillips Hue Fernbedienung verwendet, die zur Steuerung der Lampen genutzt wird.

\section{Eingesetzte Software}

\subsection{Raspbian OS}

RaspbianOS ist eine leichtgewichtige Linux Distribution, welche direkt vom Hersteller des RaspberryPis speziell auf die Bedürfnisse des Board angepasst ist. Es enthällt eine
Desktop Umgebung sowie grundlegende Pakete. Es basiert auf Debian, damit sind entsprechende gut gepflegte Paketquellen verfügbar. 

\subsection{Docker}

Docker ist eine Containerisierungsplattform, um Anwendungen containerisiert auf Linux-Servern ausführen zu können. Docker reduziert erheblich den Aufwand 
Anwendungen zu betreiben. Docker Container beinhalten alle Abhängigkeiten um die Anwendung im Container lauffähig zu machen.
Prozesse laufen in eigenen Namespaces und sind dadurch abgekoppelt von anderen Containern sowie dem Betriebssystem des Hosts. Im Unterschied zur Virtualisierung werden
einige Ressourcen gemeinsam genutzt. Dadurch ist die Effizienz höher als bei traditioneller Virtualisierung, bei der ein vollständiges Betriebssystem virtualisiert wird.

\subsection{Docker-Compose}

Docker-Compose ist ein Tool, um Containerumgebungen im Textformat, hier YAML, zu beschreiben.
Ein Container kann im einfachen Fall per Docker-CLI mit entsprechenden Parametern gestartet werden:
\begin{lstlisting}
  docker run hello-world -v ./home:/home -p 80:80
\end{lstlisting}

Durch diesen Befehl wird der Container \grqq hello-world\grqq{} aus dem Docker Repository geladen und anschließend gestartet. In diesem ist ein einfacher Webserver der bei
Aufruf ein \grqq Hello world !\grqq{} zurückgibt implementiert. Zusätzlich wird der Ordner \grqq home\grqq{} in den Container gemountet. Dieser bleibt auch bei einem
erneuten Laden des Containers persistent. Dies wird beispielweiße für Konfigurationsdateien oder andere persistente Dateien genutzt. Um den Container auch auf der Schnittstelle
des Host-Systems verfügbar zu machen, wird der Port 80 auf den Container Port 80 gebunden. Die Funktionsweise wird später erläutert.

Als handliche Alternative zu der Docker-CLI lässt sich der Zielzustand auch in einer Textdatei beschreiben:

\begin{lstlisting}
version: '3'
services:
  helloworld:
    container_name: helloworld
    image: hello-world
    ports:
      - 80:80
    volumes:
      - ./home:/home
    restart: unless-stopped
\end{lstlisting}

Mit einem 
\begin{lstlisting}
  docker-compose up -d
\end{lstlisting}

gelangt man zum selben Ergebnis wie mit dem vorher gezeigtem CLI Befehl. Durch die Beschreibung in der Textdatei ist das Ergebnis allerdings reproduzierbar.

\subsection{Zigbee2mqtt}

\begin{figure}[H]
  \centering
  \includegraphics[width=1\textwidth]{media/z2m-arch.png}
  \caption{Zigbee2Mqtt Kontext}
\end{figure}

Zigbee2mqtt ist ein quelloffenes Software basiertes Zigbee Gateway. Es übernimmt die Funktionalität, die normalerweise entsprechende
Bridges der Hersteller übernehmen. Während traditionelle Bridges, wie zum Beispiel die Phillips Hue Bridge eine REST API zur Verfügung stellen um mit entsprechenden
Apps zu kommunizieren, macht Zigbee2mqtt die Geräte per MQTT nach außen verfügbar. Auf abstrakter Ebene heißt das, dass es ein Gateway zwischen einem Zigbee Netzwerk und
einem traditionellen IPv4 Netzwerk ist. Zur Steuerung und Visualisierung lassen sich per MQTT Anwendungen wie \grqq Homeassistant\grqq{} oder \grqq OpenHUB\grqq{} oder auch entsprechende
Eigenentwicklungen einsetzen. Zigbee2mqtt greift direkt auf den cod.m ZigBee Adapter zu.

Quellcode und Dokumentation: \url{https://github.com/Koenkk/zigbee2mqtt} \\
Homepage: \url{https://www.zigbee2mqtt.io/}

Zigbee2mqtt verwaltet das Zigbee Netzwerk und ermöglicht es Drittanwendungen, die Geräte in diesem ZigBee Netz zu Steuern. Wird ein neues Device ins das Netzwerk eingefügt, 
kündigt Zigbee2mqtt das Gerät per MQTT an, und teilt der Anwendung nach erfolgreichem Interview die verfügbaren Cluster des neuen Teilnehmers mit.

Zigbee2mqtt verwaltet drei Datenbanken, welche die Funktionsweiße deutlich machen. Viele Funktionen, wie zum Beispiel die Verwaltung von Routingtabellen und
Verschlüsselung der Kommunikation sind direkt in der Hardware implementiert. Diese Funktionen lassen sich wie in der im Punkt TI CC Firmware gezeigten API steuern und abfragen.
Zigbee2mqtt verwaltet in einer eigenen Datenbank die Geräte im Netzwerk sowie deren Eigenschaften.
Folgende Datensätze finden sich in der Anwendung:

\textbf{coordinator-backup.json}\\

Hier sind die für die Initialisierung beim Start des Koordinators wichtigen Informationen abgelegt. Dies beinhaltet alle dem Netzwerk zugehörigen Geräte.
Durch löschen dieser Datei wird das Netzwerk vollständig zurückgesetzt. Die einzelnen Teilnehmer müssen dann manuell per Touchlink oder nach herstellerspezifischem Verfahren
zurückgesetzt werden, um wieder einem neuen Netzwerk beitreten zu können.


\begin{lstlisting}
  {
    "metadata": {
      "format": "zigpy/open-coordinator-backup",
      "version": 1,
      "source": "zigbee-herdsman@0.14.103",
      "internal": {
        "date": "2023-05-04T19:48:28.936Z",
        "znpVersion": 1
      }
    },
    "stack_specific": {
      "zstack": {
        "tclk_seed": "69a6670050d8354347405537724e1a81"
      }
    },
    "coordinator_ieee": "00124b0026b748c8",
    "pan_id": "1a62",
    "extended_pan_id": "00124b0026b748c8",
    "nwk_update_id": 0,
    "security_level": 5,
    "channel": 11,
    "channel_mask": [
      11
    ],
    "network_key": {
      "key": "01030507090b0d0f00020406080a0c0d",
      "sequence_number": 0,
      "frame_counter": 5552161
    },
    "devices": [
      {
        "nwk_address": "9a3b",
        "ieee_address": "0017880104b9359d",
        "is_child": false,
        "link_key": {
          "key": "87b4d0a2668847d8a876fe4454bf654e",
          "rx_counter": 0,
          "tx_counter": 121
        }
      },
  ...  
  \end{lstlisting}

  \textbf{state.json}\\

  In dieser Datei sind alle aktuellen Zustände Geräte im Netzwerk hinterlegt. Sie dient dazu, bei einem Neustart des Koordinators den letzten Zustand wieder herzustellen.

  \begin{lstlisting}
    ...
    "0xbc33acfffe9587ed": {
        "brightness": 15,
        "state": "OFF",
        "color_mode": "color_temp",
        "color_temp": 360,
        "linkquality": 40,
        "color": {
            "x": 0.4542,
            "y": 0.4092
        },
        "do_not_disturb": false
    ...
  \end{lstlisting}
  \textbf{database.db}\\

  Dies ist die zentrale Datenbank von Zigbee2mqtt. Da SQLite eingesetzt wird, lässt sich auch hier der Inhalt wie bei einer Textdatei einfach auslesen.
  Es liegt für jedes Device ein Datensatz ab.

  Datensatz des Koordinators:
  \begin{lstlisting}
{"id":1,"type":"Coordinator","ieeeAddr":"0x00124b0026b748c8","nwkAddr":0,"manufId":0,"epList":[1,2,3,4,5,6,8,10,11,12,13,47,110,242],
"endpoints":{"1":{"profId":260,"epId":1,"devId":5,"inClusterList":[],"outClusterList":[],"clusters":{},"binds":[],"configuredReportings":[],
"meta":{}},"2":{"profId":257,"epId":2,"devId":5,"inClusterList":[],"outClusterList":[],"clusters":{},"binds":[],"configuredReportings":[],
"meta":{}},"3":{"profId":260,"epId":3,"devId":5,"inClusterList":[],"outClusterList":[],"clusters":{},"binds":[],"configuredReportings":[],
"meta":{}},"4":{"profId":263,"epId":4,"devId":5,"inClusterList":[],"outClusterList":[],"clusters":{},"binds":[],"configuredReportings":[],
"meta":{}},"5":{"profId":264,"epId":5,"devId":5,"inClusterList":[],"outClusterList":[],"clusters":{},"binds":[],"configuredReportings":[],
"meta":{}},"6":{"profId":265,"epId":6,"devId":5,"inClusterList":[],"outClusterList":[],"clusters":{},"binds":[],"configuredReportings":[],
"meta":{}},"8":{"profId":260,"epId":8,"devId":5,"inClusterList":[],"outClusterList":[],"clusters":{},"binds":[],"configuredReportings":[],
"meta":{}},"10":{"profId":260,"epId":10,"devId":5,"inClusterList":[],"outClusterList":[],"clusters":{},"binds":[],"configuredReportings":[],
"meta":{}},"11":{"profId":260,"epId":11,"devId":1024,"inClusterList":[1281,10],"outClusterList":[1280,1282],"clusters":{},"binds":[],"configuredReportings":[],
"meta":{}},"12":{"profId":49246,"epId":12,"devId":5,"inClusterList":[],"outClusterList":[],"clusters":{},"binds":[],"configuredReportings":[],
"meta":{}},"13":{"profId":260,"epId":13,"devId":5,"inClusterList":[25],"outClusterList":[],"clusters":{},"binds":[],"configuredReportings":[],
"meta":{}},"47":{"profId":260,"epId":47,"devId":5,"inClusterList":[],"outClusterList":[],"clusters":{},"binds":[],"configuredReportings":[],
"meta":{}},"110":{"profId":260,"epId":110,"devId":5,"inClusterList":[],"outClusterList":[],"clusters":{},"binds":[],"configuredReportings":[],
"meta":{}},"242":{"profId":41440,"epId":242,"devId":5,"inClusterList":[],"outClusterList":[],"clusters":{},"binds":[],"configuredReportings":[],
"meta":{}}},"interviewCompleted":true,"meta":{},"lastSeen":1671278654240,"defaultSendRequestWhen":"immediate"}

  \end{lstlisting}

In dieser Datenbank sind alle Geräte, welche direkt mit dem Koordinator verbunden sind vermerkt. Zu jedem Gerät werden die gebundenen Cluster vermerkt.

\subsubsection{zigbee-herdsman}

Der Herdsman ist die eigentliche Kernanwendung von Zigbee2mqtt. Dieses Modul verbindet sich direkt über einen serielle Schnittstelle mit dem Koordinator. Über diese Schnittstelle
spricht Herdsmann die API des Koordinators an um das Netzwerk zu verwalten. Herdsman verwaltet die Datenbank und damit den Zustand des Netzwerkes. Das Modul stellt nach außen
eine API zur Verfügung, mit der sich das Netzwerk verwalten lassen kann. Auf diese API greift auch die integrierte WebGui zu.

Die API von zigbee-herdsman wird im entsprechenden GitHub Repository dokumentiert. \\
\url{https://github.com/Koenkk/zigbee-herdsman}

\subsubsection{zigbee-herdman-converters}

Dieses Modul dient zur Abstraktion verschiedenster Geräte für Zigbee2mqtt. Das Modul hat ein eigenes Repository. Es liegt für jeden Hersteller von unterstützen Smart Devices eine Typescript
Datei ab, in der die Geräte definiert werden. Die Geräte werden Anhang ihrer ID erkannt. Anschließend werden die Cluster beschrieben sowie die gültigen parametrierbaren Werte.

\begin{lstlisting}
const definitions: Definition[] = [
    {
        fingerprint: [{modelID: 'TS130F', manufacturerName: '_TZ3000_vd43bbfq'}, {modelID: 'TS130F', manufacturerName: '_TZ3000_fccpjz5z'}],
        model: 'QS-Zigbee-C01',
        vendor: 'Lonsonho',
        description: 'Curtain/blind motor controller',
        fromZigbee: [fz.cover_position_tilt, fz.tuya_cover_options],
        toZigbee: [tz.cover_state, tz.cover_position_tilt, tz.tuya_cover_calibration, tz.tuya_cover_reversal],
        meta: {coverInverted: true},
        exposes: [e.cover_position(), e.enum('moving', ea.STATE, ['UP', 'STOP', 'DOWN']),
            e.binary('calibration', ea.ALL, 'ON', 'OFF'), e.binary('motor_reversal', ea.ALL, 'ON', 'OFF'),
            e.numeric('calibration_time', ea.STATE).withUnit('S').withDescription('Calibration time')],
    },
  \end{lstlisting}

  An dieser Stelle wird ein Rolladenmotor von Lonsonho definiert. Damit kann dieser durch Zigbee2mqtt erkannt und gesteuert werden. In weiten Modulen werden dann die Werte die von den Geräten kommen
  beziehungsweiße an diese gesendet werden auf einen globalen Standard konvertiert. Damit können Rolladenmotoren von verschiedenen Herstellern mit einer Anwendung identisch gesteuert werden. Das 
  Repository findet sich unter \url{https://github.com/Koenkk/zigbee-herdsman-converters}. Durch den quelloffenen Ansatz ließe sich ein Gerät, welches bisher noch nicht unterstützt wird, hier nachpflegen.

\subsubsection{zigbee2mqtt}

Dieses Modul umschreibt die beiden vorher beschriebenen Module und umfasst noch eine Weboberfläche. Die Weboberfläche dient zur Verwaltung und Visualisierung des Netzwerkes.
\begin{figure}[H]
  \centering
  \includegraphics[width=1\textwidth]{media/z2m.png}
  \caption{zigbee2mqtt Webfrontend}
\end{figure}

Die Weboberfläche biete die Möglichkeit alle Endpunkte von Herdsman abzufragen und entsprechend zu steuern.

\begin{figure}[H]
  \centering
  \includegraphics[width=1\textwidth]{media/z2m-map.png}
  \caption{zigbee2mqtt Netzwerkvisualisierung}
\end{figure}

Das Netzwerk lässt sich in einer dynamischen Übersicht visualisieren. Hier die aktiv genutzten Verbindung zwischen den Geräten. 


\subsubsection{TI CC Firmware}

Eine Firmware für die Texas Instruments Chips, um diese als Koordinator einsetzen zu können. Die Firmware basiert auf dem Z-Stack von Texas Instruments. Sie wird fertig kompiliert
in dem Git-Repo von Zigbee2mqtt angeboten. Sie kann auf die USB-Koordinatoren per USB geflasht werden, der Einsatz eines Launchpads ist nicht notwendig. Eine Anleitung
findet sich auf der Homepage von Zigbee2mqtt.  


Zur Veranschaulichung der Funktionsweiße, ein Ausschnitt aus der API Dokumentation:

\begin{figure}[H]
  \centering
  \includegraphics[width=1\textwidth]{media/z-stack-api-excerpt.png}
  \caption{Z-Stack API Auszug}
\end{figure}

In diesem Beispiel wird beschrieben, wie man einen SimpleDescriptor-Request an ein Zigbee-Device versendet. Dieser Aufruf ist entsprechend parametrierbar,
und wird zur Abfrage der verfügbaren Endpunkte eines Gerätes nach dessen Beitritt in das Netzwerk genutzt.

Die Firmware, die auf dem Koordinator zum Einsatz kommt basiert auf einem Beispielprojekt für einen ZigBee Koordinator von Texas Instruments. Dieses ist
im SDK Kit des TI CC2652 Chips enthalten, und kann mit dem Texas Instruments Code Compose Studio bearbeitet werden. Die Firmware kann selbst kompiliert werden.
Dazu kann ein Git Patch von den Maintainern von Zigbee2mqtt angewendet werden. Zusätzlich können nun eigene Änderungen in den Code einfließen. Der Patch ist unter
folgender URL zu finden.
\url{https://github.com/Koenkk/Z-Stack-firmware/blob/master/coordinator/Z-Stack_3.x.0/firmware.patch} .
Die Änderungen, die vorgenommen werden können hier nachgelesen werden. Eine Anleitung zum kompilieren liegt ebenfalls im Git Repository ab.

\begin{figure}[H]
  \centering
  \includegraphics[width=1\textwidth]{media/cc-device.png}
  \caption{TI Code Compose Studio}
\end{figure}
In der Abbildung ist das geöffnete Projekt im Code Compose Studio mit den verschiedenen Konfigurationen die vorgenommen werden können.

\subsection{MQTT}

MQTT \cite{mqtt} ist ein Protokoll, um Nachrichten zwischen Teilnehmen in einem IP Netzwerk auszutauschen. Alle Nachrichten werden unter einem definierten Topic
an einen zentralen Messagebroker gesendet. Teilnehmer können Topics abonnieren. Der Broker verwaltet eine Liste mit allen Teilnehmern sowie 
deren abonnierten Topics. Wird eine entsprechende Nachricht an den Broker gesendet, werden alle Abonnenten des Topics entsprechend informiert.
In dem Versuch wird der quelloffene MQTT Broker \grqq mosquitto\grqq{} eingesetzt. 

\subsection{Wireshark}

Wireshark ist eine quelloffene Anwendung um Datenströme mitzuschneiden und zu untersuchen. Wireshark selbst nutzt standardmäßig \grqq npcap\grqq{} um Datenverkehr 
auf Netzwerkkarten aufzuzeichnen. Es ist möglich über andere Schnittstellen Wireshark Datenströme zur Verfügung zu stellen. Zu diesem Zweck
können ausführbare Dateien in einen Ordner \grqq .../extcap\grqq{} abgelegt werden, welche Paketströme zurückliefern. 

\begin{figure}[H]
  \centering
  \includegraphics[width=1\textwidth]{media/wireshark.png}
  \caption{Wireshark}
\end{figure}

\subsection{Ansible}

Ansible ist ein Werkzeug zur Automatisierung. Arbeitsabläufe lassen sich strukturiert in YAML (yet another markup language) definieren. Dies ist eine alternative zum schreiben
von Shell Scripten. Ansible kann Aufgaben auf dem lokalem System und Remote ausführen. Aufgaben können in Rollen zusammengefasst werden.  
Eine Rolle kann einem Host wie folgt zugewiesen werden:
\begin{lstlisting}
  - name: Deploy the Lab
    hosts: localhost
    roles:
    - DeployDocker
    - DeployLabUtils
\end{lstlisting}

Die Rollen \grqq DeployDocker\grqq{} und \grqq DeployLabUtils\grqq{} umfassen eine Menge von Aufgaben zur Installation notwendiger Komponenten und weitere Vorbereitungen
für den Praktikumsversuch. Diese Rollen werden \grqq localhost \grqq{}, also dem ausführendem System selbst zugewiesen.
Aus technischer Sicht lädt Ansible parametrierte Pythonmodule auf den jeweiligen Host per SCP und führt diese dort aus.


