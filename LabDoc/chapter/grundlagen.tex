\chapter{Grundlagen}

In diesem Kapitel werden alle verwendeten Komponenten und Technologien kurz erläutert.

\section{LR-WPAN - IEEE 802.15.4}

LR-WPAN steht für \grqq Low Rate - Wireless personal area network \grqq{}. Es handelt sich um ein drahtloses geschlossenes Netzwerk, welches für 
niedrige Datenraten ausgelegt ist. Der Standard definiert den Physical Layer sowie den Media-Access Layer und ist damit die Grundlage von ZigBee, Thread und 6LowPAN.
Im Standard sind mehrere Modulationsverfahren sowie Frequenzbereiche definiert. Peer-to-Peer ist Teil des Standards. Im vergleich zum 802.1d Standard fallen 
vorallem die kürzeren Adressen auf. Dadurch kann die für den Anwendungsbereich wertvolle Bandbreite und Rechenleistung reduziert werden. 

\section{ZigBee}

Die ZigBee Alliance wurde durch ein Konsortium von Herstellern gegründet, um einen einheitlichen Übertragungstandard
im Bereich Heimautomatisierung voranzubringen. ZigBee basiert auf dem offenem 802.15.4 Standard, bringt allerdings Komponenten mit die nicht in einem IEEE
Standard definiert sind.
ZigBee ist in Form von weiteren Protokollschichten implementiert, welche auf IEEE 802.15.4 aufsetzen. ZigBee nutzt DSSS, also Frequenzspreizung als Modulationsverfahren.
Die genutzten Kanäle, 11 bis 26, liegen im 2,4 Ghz Band. Zigbee interferiert damit mit WLAN.

\begin{figure}[H]
  \centering
  \includegraphics[width=1\textwidth]{media/Zigbee Stack.jpg}
  \caption{ZigBee Protocoll Stack \\ Bildquelle: \url{https://www.researchgate.net/figure/IEEE820154-ZigBee-protocol-stack-architecture_fig2_265150617}}
\end{figure}

Der Anwendungsbereich für ZigBee ist die Heimautomatisierung. Geräte, hier als Devices bezeichnet, können zentral gesteuert und überwacht werden. 
Markanter Vorteil von ZigBee ist, dass die Geräte keine direkte Funkverbindung
zu einem zentralen Controller brauchen. Einzelne Geräte können als Router fungieren, und damit die Reichweite erhöhen. Dies sorgt dafür,
dass mir einer geringen Sendeleistung ein geografisch großes Gebiet abgedeckt werden kann. Sende- und Empfangsleistung
ist vorallem bei kleinen Batteriebetriebenen Geräten oft der einschränkende Faktor.


\section{Texas Instruments CC Chips}

Texas Instruments bietet ein Spektrum von Microcontrollern, die sich mit entsprechender Firmware für ZigBee Geräte 
nutzen lassen können. Kleinere Chips können in Endgeräten wie Lampen und Thermostate, größere als Koordinator selbst verwendet werden.

Die aktuelle Chipfamilie TexasInstruments CC26XX:

\begin{figure}[H]
  \centering
  \includegraphics[width=1\textwidth]{media/table26xx.png}
  \caption{Test des Messagebrokers Mosquitto}
\end{figure}

Als Koordinator werden die Leistungsfähigeren Chips aus der 265X Reihe eingesetzt. ZigBee Geräte nutzen in einigen
Fällen Bluetooth LE zur Koppelung, daher ist die Unterstüzung diesen Protokolls sinnvoll.

\begin{figure}[H]
  \centering
  \includegraphics[width=1\textwidth]{media/table265x.png}
  \caption{TI CC 265X Serie}
\end{figure}

In der Tabelle sind die unterstützten Protokolle der einzelnen Modelle sowie deren Leistungsfähigkeit aufgeführt.
Es ist anzumerken, dass die größeren Modelle schon den Standard Thread unterstützen, der vermutlich durch das
Projekt \grqq Matter\grqq{} erheblich an bedeutung gewinnen wird.

Texas Instruments stellt als Basis für ZigBee Anwendungen eine Z-Stack Bibliothek zur Verfügung. Diese stellt alle grundlegenden 
Funktionen um das ZigBee Protokoll zu implementieren. Mit Texas Instruments Code Composer Studio steht eine IDE bereit,
um den Entwicklungsprozess zu unterstützen. Auf den entsprechend Leistungsfähigeren Chips lassen sich in freie Speicherbereiche noch zusätzliche
Funktionalitäten einprogrammieren. Die Chips können mit Programmierboards des Herstellern programmiert werden. Alternativ kann man
günstig einen USB-Stick mit aufgelöteten CC Chip erwerben, und auch diesem mit entsprechenden Tools programmieren.

Weiter Informationen: \url{https://www.ti.com/tool/Z-STACK#overview}

In dem OpenSource Projekt "zigbee2mqtt" werden ausschließlich Chips von Texas Instruments unterstützt. Die meißten gängigen Anbieter von Microchips 
haben entsprechende Modelle im Angebot. 

\section{Versuchshardware}

\subsection{RaspberryPi}

Der RaspberryPi ist ein ARM basierter Computer im Mini-Format. Er dient in diesem Versuch als Server, der die Applikationen
hostet und gleichzeitig als Versuchs-PC, auf dem der Versuch durchgeführt wird. Die eingesetzten Anwendungen sind 
als Webservice implementiert und werden per Docker Containerisierung ausgerollt.

Der RaspberryPi besitzt die PC typtischen Schnittstellen wie Ethernet, HDMI, sowie USB. Als Festspeicher wird eine
SD-Karte eingestzt. 

Auf dem RaspberryPi wird das Linux-basierte Betriebssystem RaspbianOS. Dies ist eine von den Entwicklern des RaspberryPis eigenentwickelte und für den RaspberryPis
angepasste Linux-Distribution. Es baut auf Ubuntu auf.

\subsection{RaspberryPi Zigbee Hat}

Als Zigbee Koordinator wird ein auf dem TI CC2652 basierendem RaspberryPi Hat vom Hersteller \grqq cod.m \grqq{} eingesetzt. Dieser wurde vom Hersteller
für den Einsatz mit \grqq homegear \grqq{} oder \grqq zigbee2Mqtt \grqq{} entwickelt. Ein Datenblatt sowie ein Manual ist im Anhang.

\subsection{CC2531 Sniffer Stick}

Mit diesem Stick wird die ZigBee Kommunikation in dem Netzwerk des Versuchaufbaufs mitgeschnitten.
Der stick basiert auf einem leistungsschwachen Chip, der mit entsprechender \grqq Snifffer \grqq{} Firmware Pakete \grqq over the air \grqq{} mitschneidet.
Die Pakete lassen sich über die serielle USB-Schnittstellen im Anschluss auslesen, und in Wireshark visualisieren. 

Als Treiber wird ein in C geschriebenes Programm verwendet, welches es ermöglicht den Stick direkt als Interface in Wireshark hinzuzufügen. Der Quellcode findet sich in 
GitHub unter https://github.com/andrebdo/wireshark-cc2531. Hier ist auch eine Anleitung zum kompilieren. Die hieraus entstehende ausführbare Datei muss in entsprechenden
Wireshark Ordner kopiert werden, und kann anschließend als Interface ausgewählt werden. Die Funktion nennt sich bei Wireshark \grqq extcap \grqq{}.

todo: Screenshot wireshark

\subsection{Phillips Hue Komponenten}

Die Phillips Hue Serie setzt vollständig auf den ZigBee Standard. Die Produktlinie umfasst alle Komponenten zur Heimautomatisierung. Dies umfasst eine APP,
ein Gateway sowie vielfältige Endgeräte. Sie sind Kompatibel zu dem Software-Gateway zigbee2mqtt.
Die Lampen werden in dem Versuch als Demonstrationsobjekte eingesetzt. Sie können Ein- und Ausgeschaltet werden, sowie gedimmt werden. Zusätzlich wird einer
Phillips Hue Fernbedienung verwendet, die zur Steuerung der Lampen genutzt wird. Phillips verwendet bei kleinen Geräten die ZigBee Chips von Atmel, bei größeren 
sind Eigenentwicklungen zu finden.

\section{Eingesetzte Software}

\subsection{Raspbian OS}

RaspbianOS ist eine leichtgewichtige Linux Distribution, welche direkt vom Hersteller des RaspberryPis speziell auf die Bedürfnisse des Board angepasst ist. Es enthällt eine
Desktop Umgebung sowie die grundlegende Paketen. Es basiert auf Debian, damit sind auch die entsprechenden Paketquellen verfügbar. Durch die große Verbreitung 
sind viele Anwendungen in den standard Paketquellen verfügbar.

\subsection{Docker}

Docker ist eine Container Umgebung, um Anwendungen containerisiert auf Linux-Servern ausführen zu können. Docker reduziert erheblich den Aufwand 
Anwendungen zu betreiben. Alle Abhängigkeiten sind im Container enthalten, sodass hier keine Komplikationen mit anderen Anwendungen
zu befürchten sind oder Abhängigkeiten fehlen. Prozesse laufen in eigenen Namespaces und sind dadurch abgekoppelt vom Host Betriebssystem. Im Unterschied zur Virtualisierung werden
Systemprozesse gemeinsam genutzt. Dadurch ist die Effizienz höher als bei taditioneller Virtualisierung.

\subsection{Docker-Compose}

Docker-Compose ist ein Tool, um große Containerumgebungen per YAML zu definieren und automatisch deployen. Klassisch kann ein Container 
per CLI Befehl mit entsprechenden Parametern gestartet werden:
\begin{lstlisting}
  docker run hello-world -v ./home:/home -p 80:80
\end{lstlisting}
Start des Container \grqq hello-world \grqq{}, es wird ein Container gemountet und der Port 80 auf den Host gemappt.
Dies ist allerdings für große Umgebungen unhandlich. Alternativ werden Container als Services in einer YAML definiert.

\begin{lstlisting}
version: '3'
services:
  helloworld:
    container_name: helloworld
    image: hello-world
    ports:
      - 80:80
    volumes:
      - ./home:/home
    restart: unless-stopped
\end{lstlisting}

Mit einem 
\begin{lstlisting}
  docker-compose up -d
\end{lstlisting}

können dann alle definierten Services gestartet werden. Dies folgt der allgemeinen Konvention "Konvention vor Konfiguration". Der Zielzustand wird einmal definiert
und kann Aufgrund dieser Definition beliebig oft ausgerollt werden.

\subsection{zigbee2mqtt}

zigbee2mqtt ist ein offenes Softwareprojekt und kann als \grqq Software-Zigbee-Gateway \grqq bezeichnet werden. Es übernimmt die Funktionalität, die normalerweise entsprechende
\grqq Bridges \grqq der Hersteller übernehmen. Während traditionelle Bridges, wie zum Beispiel die Phillips Hue Bridge eine REST API zur Verfügung stellen um mit Ihren entsprechenden
Apps zu kommunizieren, macht zigbee2mqtt die Geräte per mqtt nach außen verfügbar. Auf abstrakter Ebene bedeutet dies, das es ein Gateway zwischen einem Zigbee Netzwerk und
einem traditionellen IPv4 Netzwerk ist. Zur Steuerung und Visualisierung lassen sich per MQTT Anwendungen wie \grqq Homeassistant \grqq oder \grqq OpenHUB \grqq{} bzw. entsprechende
Eigenentwicklungen einsetzen.

Quellcode und Dokumentation: \url{https://github.com/Koenkk/zigbee2mqtt}
Homepage: \url{https://www.zigbee2mqtt.io/}

zigbee2mqtt verwaltet das Zigbee Netzwerk und ermöglicht es dritten Anwendungen, die Geräte in diesem ZigBee Netz zu Steuern. Wird ein neues Device ins das Netzwerk eingefügt,
kündigt zigbee2mqtt das Gerät per MQTT an, und gibt nach erfolgreichem Interview alle Cluster an.

todo: Screenshot MQTT

Anschließend können Drittanwendungen, die die entsprechende Schnittstelle implementiert haben, die Geräte ebenfalls per MQTT steuern.

Zigbee2Mqtt verwaltet 3 persistente Datensätze, welche die Funktionsweiße deutlicher machen. Viele Protokollnahe Funktionen, wie zum Beispiel das Pflegen von Routingtabellen und
Verschlüsselung der Kommunikation sind direkt in der Hardware implementiert. Diese Funktionen lassen sich wie in der im Punkt TI CC Firmware gezeigen API Steuern und Abfragen.
Zigbee2mqtt verwaltet in einer eigenen Datenbank die Geräte im Netzwerk sowie deren Eigenschaften.
Folgende Datensätze finden sich in der Anwendung:

\textbf{coordinator-backup.json}\\

Wie der Name sagt, sind hier die für die Initialisierung beim Start des Koordinators wichtigen Informationen abgelegt. Dies beinhaltet alle dem Netzwerk zugehörigen Geräte.
Durch löschen dieser Datei wird das Netzwerk vollständig zurückgesetzt. Die einzelnen Teilnehmer müssen nun aber manuell per Touchlink oder nach herstellerspezifischem Verfahrem
zurückgesetzt werden, um wieder einem neuen Netzwerk beitreten zu können.
Der Inhalt dieser Datei sieht wie folgt aus:
\begin{lstlisting}
  version: '3'
  services:
    helloworld:
      container_name: helloworld
      image: hello-world
      ports:
        - 80:80
      volumes:
        - ./home:/home
      restart: unless-stopped
  \end{lstlisting}

  \textbf{state.json}\\

  In dieser Datei sind alle aktuellen Zustände Geräte im Netzwerk hinterlegt. Sie dient dazu, bei einem Neustart des Koordinators den alten Zustand wieder herzustellen.
  Eine Abfrage eines Attributes der Geräte sollte immer über die API erfolgen.

  Der Inhalt sieht wie folgt aus:

  \textbf{state.json}\\

  Dies ist die eigentliche Datenbank von zigbee2mqtt. Da SQLite eingesetzt wird, lässt sich auch hier der Inhalt wie bei einer Textdatei einfach auslesen. Hier sind alle
  Konfigurationen der Anwendung, aber auch für jedes Device sämtliche Informationen hinterlegt.

  Auszug aus dem Konfigurationsteil:
  \begin{lstlisting}
  \end{lstlisting}

  Ein Datensatz eines Devices



\subsubsection{TI CC Firmware}

Firmware für die Texas Instruments Chips, um diese als Koordinator einsetzen zu können. Die Firmware basiert auf dem Z-Stack von Texas Instruments. Sie wird fertig kompiliert
in dem Git-Repo von zigbee2mqtt angeboten. Sie kann auf die USB-Koordinatoren direkt per USB geflasht werden, der Einsatz eines Launchpads ist nicht zwingend notwendig. Eine Anleitung
findet sich auf der Homepage von zigbee2mqtt.  


Zur veranschaulichung, hier ein Ausschnitt aus der API Dokumentation.

\begin{figure}[H]
  \centering
  \includegraphics[width=1\textwidth]{media/z-stack-api-excerpt.png}
  \caption{Z-Stack API Auszug}
\end{figure}

In diesem Beispiel wird beschrieben, wie man einen SimpleDescriptor-Request an ein Zigbee-Device versendet. Dieser Aufruf ist entsprechend Parametrierbar,
und wird zur Abfrage der verfügbaren Endpunkte eines Gerätes nach dessen Beitritt in das Netzwerk abgefragt.

\subsubsection{zigbee-herdsman}

Der Herdsman ist die eigentliche Kernanwendung von zigbee2mqtt. Diese Modul verbindert sich direkt über einen seriellen Socket mit dem Koordinator. Über diese Schnittstelle
spricht Herdsmann die API des Koordinators an um das Netzwerk zu verwalten. Herdsman verwaltet die Datenbank und damit den Zustand des Netzwerkes. Das Modul stellt nach außen
eine API zur Verfügung, mit der Sich das Netzwerk verwalten lassen kann. Auf diese API greift auch die integrierte WebGui zu.

Die API von Herdman wird im entsprechenden GitHub Repository dokumentiert.
\url{https://github.com/Koenkk/zigbee-herdsman}

\subsubsection{zigbee-herdman-converters}

Dieser Konverter kann proprietäre Cluster die von selbstentwickelten Devices oder manch Devices von Drittherstellern. Mit diesem Converter lassen sich proprietäre Cluster von Geräte
so adaptieren, dass sie nach Wunsch gesteuert und ausgelesen werden können.

\subsubsection{zigbee2mqtt}

Dieses Modul umfasst die beiden vorher beschriebenen Module und fügt noch eine WebGUI hinzu. Die WebGUI dient zur einfachem Verwaltung und Visualisierung des Netzwerkes.
\begin{figure}[H]
  \centering
  \includegraphics[width=1\textwidth]{media/z2m.png}
  \caption{zigbee2mqtt Webfrontend}
\end{figure}

Die WebGUI enthält eine große Anzahl von Funktionen, die weitaus tiefer reichen als für die Nutzung notwendig sind.
Prinzipiell sind die meißten ZigBee Geräte Kompatibel, wenn ein Community Mitglied dieses bereits in der Anwendung
angelegt hat. Es ist auch möglich, eigene Beschreibungen für nich nicht unterstützte Geräte zu erstellen.\\

\begin{figure}[H]
  \centering
  \includegraphics[width=1\textwidth]{media/z2m-map.png}
  \caption{zigbee2mqtt Netzwerkvisualisierung}
\end{figure}

Das Netzwerk lässt sich in einer dynamischen Übersicht visualisieren. Hier die aktiv genutzen Verbindung zwischen den Geräten. 



\subsection{Wireshark}

Wireshark ist eine quelloffene Anwendung um Datenstöme Mitzuschneiden und zu Untersuchen. Es kann durch Verwendung
von Packetsniffern wie nPcap verschiedenste Medien wie zum Beispiel Ethernet und USB mit entsprechenden Protokollen
verarbeiten.

\todo{Screenshot Wireshark mit Zigbee Paketen}

\subsection{Ansible}

Ansible ist ein Tool zur IT-Automatisierung. Arbeitsabläufe lassen sich ebenfall gut strukturiert in YAML definieren. Ansible verbindet sich per SSH 
mit den zu konfigurierenden Hosts und führt Python Module aus. Hosts lassen sich in Gruppen, und Tasks in sogenannten Rollen sammeln. So ist es möglich,
eine Gruppe von Hosts die Rolle \grqq Webapplikation \grqq{}, und einer Gruppe von anderen Hosts der Gruppe \grqq Datenbank \grqq{} zuweisen. 
Eine Zuweisung von Rollen sihet folgendermaßen aus:
\begin{lstlisting}
  - name: Deploy the Lab
  hosts: localhost
  roles:
    - DeployDocker
    - DeployLabUtils
\end{lstlisting}

In diesem Beispiel wird auf dem lokalen Host Docker installiert, und das ZigBee Praktikum vorbereitet.

\subsection{}
