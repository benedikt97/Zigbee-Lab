\chapter{Einführung}

In diesem Projekt soll ein Praktikumsversuch für die Vorlesung Internet of Things für Professor Dr. Jürgen Winter 
entwickelt werden. In dem Versuch soll die FUnktionsweise des Funkprotokolls ZigBee untersucht werden. Es wird ein kleines
ZigBee Netz mit mehreren Teilnehmen aufgebaut und die Kommunikation zwischen diesen aufgezeichnet und untersucht. 
Es wird eine Versuchsanleitung entwickelt, die Schritt für Schritt durch den Versuch führt.

\section{Anforderungen an die Praktikumsarbeit}

Die Anforderungen an der Versuch werden an dieser Stelle definiert, um in dieser Dokumentation 
darauf Bezug nehmen zu können.
\begin{itemize}
    \item \textbf{A010} - Der Versuch soll an einem Tag durchführbar sein.
    \item \textbf{A020} - Der Versuch soll kein Vorwissen in Linux vorraussetzen
    \item \textbf{A030} - Der Versuch setzt Vorwissen in Paketorientierten Datenübetragung vorraus.
    \item \textbf{A040} - Der Versuch setzt Vorwissen in der Bedienung von Wireshark vorraus.
    \item \textbf{A050} - Der Versuch soll zu Hause und in der Hochschule durchführbar sein.
    \item \textbf{A100} - Der Versuch soll automatisch auf den Raspberry ausgerollt werden können.
    \item \textbf{A120} - Es soll aktuelle und quelloffene Software zum Einsatz kommen.
    \item \textbf{A210} - Es soll die Funktionsweise des Joinings, des Routings, des Bindings sowie der Gruppenbildung untersucht werden.
    \item \textbf{A210} - Es sollen die implementierten Sicherheitsmechanismen untersucht und bewertet werden.
\end{itemize}

