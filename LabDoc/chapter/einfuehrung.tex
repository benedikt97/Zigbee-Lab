\chapter{Einführung}

In diesem Projekt soll ein Praktikumsversuch für die Vorlesung Internet of Things für Professor Jürgen Winter 
entwickelt werden. In dem Versuch soll das Verhalten des ZigBee Protokolles untersucht werden. Es wird ein kleines
ZigBee Netz mit einigen wenigen Komponenten aufgebaut und die Kommunikation mitgeschnitten. Anschließend wird die
Kommunikation mit dem Protokollanalyse Werkzeug Wireshark untersucht. Es wird ein \grqq LabGuide \grqq{} geschrieben,
dass Schritt für Schritt durch den Versuch führt.

\section{Anforderungen an die Praktikumsarbeit}

Die Anforderungen an der Versuch werden an dieser Stelle definiert, um in dieser Dokumentation 
darauf Bezug nehmen zu können.
\begin{itemize}
    \item \textbf{A010} - Der Versuch soll an einem Tag durchführbar sein.
    \item \textbf{A020} - Der Versuch soll kein Vorwissen in Linux vorraussetzen
    \item \textbf{A030} - De Versuch setzt Vorwissen in Paketorientierten Datenübetragung vorraus.
    \item \textbf{A040} - Der Versuch setzt Vorwissen in der Bedienung von Wireshark vorraus.
    \item \textbf{A050} - Der Versuch soll zu Hause und in der Hochschule durchführbar sein.
    \item \textbf{A060} - Studenten sollen eine Versuchsbeschreibung sowie alle nötigen Utensilien erhalten. Im Heimversuch müssen KVM-Komponenten von den Studenten selbst gestellt werden.
    \item \textbf{A100} - Der Versuch soll automatisch auf den Raspberry ausgerollt werden können.
    \item \textbf{A110} - Es wird, bis auf den Ausrollvorgang, keine Internetverbindung benötigt.
    \item \textbf{A120} - Es soll ausschließlich gewartete und quelloffene Software zum Einsatz kommen.
    \item \textbf{A200} - Der Versuch soll die Grundlagen eines Mesh-Netwerkes vermitteln.
    \item \textbf{A210} - Es soll die Funktionsweise des Joinings, des Routings, des Bindings sowie der Gruppenbildung untersucht werden.
    \item \textbf{A210} - Es sollen die implementierten Sicherheitsmechanismen untersucht und bewertet werden.
\end{itemize}

