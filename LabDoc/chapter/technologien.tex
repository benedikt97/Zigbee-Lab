\chapter{Marktübersicht Technologien}

\grqq Internet of things\grqq{} beschreibt die Befähigung von Endgeräten, Maschinen und Sensoren mit Datennetzen zu kommunizieren. So können Waschmaschinen einen fertigen Waschgang kommunizieren,
oder ein Heizungsthermostat die Außentemperatur aus dem Wetterbericht beziehen. Es lassen sich also Daten von den Geräten sammeln und die Geräte auch steuern. Im Heimbereich 
wird die Vernetzung von Geräten zur Automatisierung von zum Beispiel Steckdosen und Lampen verwendet.
Viele Hersteller haben mittlerweile ein breites Portfolio an sogenannten \grqq Smart Devices\grqq{} und der dazu
je nach Übertragungsprotokoll notwendigen \grqq Bridge\grqq{} sowie einer entsprechenden App zur Steuerung. Der Hersteller Phillips mit seiner Produktmarke \grqq Hue\grqq{}
kann als Beispiel genannt werden. Die Produktgruppe umfasst eine Bridge mit zugehöriger Smartphone App, sowie den klassischen Komponenten wie Lampen, Steckdose und Schalter.
Der Markt wurde durch Heimassistenten stark belebt. Amazons Alexa, der Google Echo Dot und die Pendanten von Apple und Microsoft sind in immer mehr Haushalten zu finden.
Diese Heimassistenten können sich entweder mit den herstellerspezifischen Bridges verbinden, oder können direkt an PANs (Personal-Area-Networks) wie Zigbee
teilnehmen und die Geräte steuern. Dieses Kapitel soll einen Überblick über die relevanten Technologien am Markt verschaffen. 

\section{Funkprotokolle}
Aktuell gibt es mehrere Funkprotokolle, welche im Bereich IoT relevant sind. Dazu gehören:
\begin{itemize}
    \item \textbf{Wlan} \\
    Wlan ist ein verbreiteter und etablierter Standard, der überwiegend für die Anbindung mobiler Geräte an den
    Internetrouter genutzt wird. Dies macht es naheliegend, auch smarte Geräte per WLAN zu vernetzen. Wlan ist allerdings 
    optimiert für hohe Übertragungsraten und nicht für leistungsschwache Endgeräte. Dies ist
    insbesondere für batteriebetriebene Geräte nachteilig. Zusätzlich benötigt jedes Endgerät Zeitslots um Daten zu Senden und zu Empfangen. Diese müssten auch für Geräte freigehalten werden, die eigentlich
    sehr wenig Bandbreite benötigen. Viele Geräte können sich also nachteilig auf die Performance eines Wlan Netzwerks auswirken. Dieses Problem ist seit WIFI6 mit OFDMA durch mehrere Subcarrier 
    zumindest weniger stark ausgeprägt.

    \item \textbf{Bluetooth}\\
    Ebenso wie Wlan hat Bluetooth eine weite Verbreitung. Durch Implementierung 
    des Standard Bluetooth LE ist es möglich leistungsschwache sowie batteriebetriebene Geräte mit Bluetooth auszustatten. Bluetooth
    ist allerdings nicht für hohe Reichweiten oder für Netzwerke mit vielen Teilnehmern konzipiert. Primäre Anwendungsfall ist zum Beispiel das Verbinden eines Headsets mit 
    einem Handy. Seit 2017 existiert der Standard \grqq Bluetooth Mesh\grqq. Dieser löst Probleme des normalen Bluetooth um es für IoT einzusetzen. Bisher gibt es keine bekannten
    größeren Ecosysteme in denen es eingesetzt wird. Ein Benchmark zeigt keine Vorteile gegenüber ZigBee und Thread. \cite{sila}

    \item \textbf{Z-Wave}\\
    Z-Wave \cite{zwave} ähnelt technologisch Zigbee, basiert allerdings nicht auf dem IEEE 802.15.4 Standard. Das Protokoll ist proprietär. Der Hauptunterschied 
    ist, dass Z-Wave in einem frei nutzbaren Low-Frequency Band arbeitet und nicht wie ZigBee im 2,4 Ghz Band. Die Reichweite ist durch die geringere 
    Trägerfrequenz höher. Die Verbreitung am Markt ist allerdings geringer.

    \item \textbf{ZigBee}\\
    Zigbee \cite{zigbee} ist ein auf den 802.40.5 Standard aufbauendes Protokoll, welches für die Anbindung vieler leistungsschwacher
    Geräte in einem großen räumlichen Areal konzipiert ist. Ein großer konzeptioneller Vorteil ist, dass bei 
    ZigBee ein Mesh-Netzwerk aufgebaut wird. Es können auch Geräte angebunden werden, die keine direkte Funkverbindung
    zum Koordinator haben. Zusätzlich sind Funktionen implementiert, welche das Management einer hohen Anzahl von Devices
    erleichtert. Ein großer Nachteil der bei ZigBee oft angeführt wird ist, dass die Geräte nicht direkt per IPv6 adressierbar sind. 2013 wurde 
    ein Standard mit mit dem Namen \grqq ZigBee IP\grqq{} veröffentlicht, der IPv6 unterstützt. Auf diesem wiederum basiert ZigBee Smart Energy, welches in 
    intelligenten Stromzählern eingesetzt wird. In Großbritannien sind diese bereits im Einsatz\cite{se}, in Deutschland sollen zumindest bis 2032 die alten Stromzähler 
    durch vernetzte ersetzt werden. \cite{bmwk}. Insgesamt gibt es neben Zigbee IP noch die beiden Erweiterungen ZigBee Pro und ZigBee RF4CE. Diese Erweiterungen spielen
    bei der Heimautomatisierung allerdings keine Rolle.
    
    \item \textbf{Thread}\\
    Thread \cite{thread} ist ein Funkprotokoll welches ebenfalls auf dem 802.15.4 Standard basiert. Ebenso wie ZigBee ist es Meshfähig. Ein
    entscheidendes Unterscheidungsmerkmal ist allerdings, dass die Geräte per IPv6 adressiert werden. Daher sind die Geräte
    theoretisch ohne die Verwendung einer Bridge aus einem herkömmlichen Ethernet Netzwerk adressierbar. Dies ist vor allem für die Anbindung 
    von Geräten im öffentlichem Raum wie Parkuhren vorteilhaft. Bisher setzt nur Google mit seinem Heimassistenten Nest im größeren kommerziellen
    Umfeld Thread ein. Es gibt erste Hersteller wie eve und nanoleaf welche entsprechende Smart Devices anbieten. Eine Liste mit zertifizierten Produkten ist 
    hier zu finden: \url{https://www.threadgroup.org/What-is-Thread/Thread-Benefits#certifiedproducts}
\end{itemize}

\section{Zigbee Anwendungen}

Hier werden nur Anwendungen für die den Heimanwender behandelt. Zigbee wird zusätzlich in einigen speziellen professionellen Anwendungen eingesetzt.

\subsection{Kommerzielle Anwendungen}

\subsubsection{Amazon Echo}
    Der Heimassistent Amazon Echo ab Generation 4 ist der einzige seiner Art der eine Zigbee Integration hat und damit als Gateway und Koordinator fungieren
    kann. Die Pendanten der Firmen Google, Microsoft und Apple benötigen ein dediziertes Zigbee Gateway.

\subsubsection{Phillips Hue}
    Phillips vertreibt unter dem Namen eine Zigbee Bridge und eine Vielzahl von Devices aus dem Segment Beleuchtung und Steckdosen. 

\subsubsection{Dresden Electronic}
    Dresden Electronic bietet Software und Hardware zum Aufbau von Zigbee Netzwerken an. Es werden Zigbee USB Adapter und RaspberryPi Hats mit ATMega Chips angeboten,
    sowie eine Steuerungssoftware \grqq deCONZ \grqq{}. Als komplette Produktlinie für den Endanwender gibt es die Produktsparte
    \grqq Phoscon\grqq{}, hauptsächlich zur smarten Beleuchtung.

\subsubsection*{Weitere Hersteller}
Weitere bekannte Hersteller/Marken mit Zigbee Devices und Gateways:
\begin{itemize}
    \item \textbf{Telekom} - QIVICON
    \item \textbf{Logitech} - Harmony Hub
    \item \textbf{LIDL} - Silvercrest
    \item \textbf{TUYA} - Smart Life
    \item \textbf{Innr} - ZigBee Bridge
    \item \textbf{SONOFF} - Günstige Hardware jeder Art
    \item \textbf{homee} -  modular Smart Home Central
    \item \textbf{Osram} - Lightify
    \item \textbf{Ledvance} - Zigbee fähige Steckdosen und Lampen \grqq Smart+ \grqq{}
\end{itemize}

Nachteil dieser Lösungen ist die Kompatibilität zu Geräten von Drittherstellern, welche vollständig in der Hand des Herstellers ist. In der Regel ist aus
wirtschaftlichen Gründen die Unterstützung konkurrierender Hersteller nicht gewünscht. Es ist schwierig, bei Anschaffung eines dieser Systeme die Kompatibilität
anderer Geräte sicherzustellen, da offiziell meist nur die Geräte aus dem eigenem Haus unterstützt sind.

\subsection{Nicht kommerzielle Anwendungen}

Vorteil von quelloffenen Anwendungen ist, dass diese durch eine Community gepflegt und Geräte von drittherstellern beliebig integriert werden können.
Grundlegend ist der Zigbee Standard universell, und die Kompatibilität von Geräten verschiedener Hersteller möglich.

\subsubsection{zigbee2mqtt}

Zigbee2mqtt \cite{z2m} ist ein quelloffenes Projekt auf GitHub. Die Anwendung kann anstelle von proprietären Bridges als ZigBee-Gateway
eingesetzt werden. Die Anwendung kann ein ZigBee Netzwerk in der Rolle des Koordinators verwalten und Funktionen per MQTT nach außen verfügbar machen.

\subsubsection{ZHA}
ZHA ist ein direkt in Homeassistant integriertes Plugin, um Zigbee Koordinatoren direkt in Homeassistant einzubinden. Vorteil
von ZHA ist, dass ZigBee Chips mehrerer Hersteller unterstützt werden. ZHA unterstützt neben Texas Instruments auch Hardware von Dresden Elektronik,
Silicon Labs, DIGI und ZiGate. ZHA ist für den Anwender komfortabler als Zigbee2mqtt, es sind wenige technische Informationen ersichtlich oder konfigurierbar.
Die Endgeräte Kompatibilität ist derzeit schlechter als bei Zigbee2mqtt.







