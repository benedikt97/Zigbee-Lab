\chapter{Marktübersicht Technologien}

IoT für den Heimanwender befindet sich größtenteils in der Hand etablierter Technologie-Unternehmen. Diese entwickeln meißt ein eigenes Öko-System,
welcher mehr oder weniger Kompatibel mit Produkten anderer Hersteller sind. Hier ist der Hersteller Phillips mit seiner Produktmarke \grqq Hue \grqq{}
als gutes Beispiel zu nennen. Die Produktgruppe umfasst eine Bridge mit zugehöriger App, sowie den klassischen Komponenten wie Lampen, Steckdose und Schalter.
Der Markt wurde durch sogenannte Heimassistenten stark belebt. Diese zumeißt sprachgesteuerten Geräte erfreuen sich starker beliebtheit. Zu diesen Geräten
gehört beispielweiße Amazon Alexa. Diese Heimassistenten verfügen Anbindungen zu den proprietären Gatewys, oder können teilweise sogar direkt an PANs wie Zigbee
teilnehmen und die Geräte direkt steuern. Dieses Kapitel soll einen Überblick über die Technologien am Markt verschaffen. 

\section{IoT Funkprotokolle}
Aktuell gibt es mehrere Funkprotokolle, welche im Bereich IoT relevant sind. Dazu gehören:
\begin{itemize}
    \item \textbf{Wlan}
    Wlan ist ein weit verbreiteter und etablierter Standard, der überwiegend für die Anbindung mobiler Geräte an den
    Internetrouter dient. Dies macht es naheliegend, auch smarte Geräte per WLAN einzubinden. Wlan ist allerdings 
    optimiert für hohe Übertragungsraten, und nicht für leistungsschwache Endgeräte. Die ist
    gerade für Batteriebetriene Geräte ein Nachteil. Zusätzlich ist es oft nicht gewünscht, Smarte Devices
    an ein Netzwerk mit Internetzugang anzuschließen, da diese sobald sie mit dem Internet kommunizieren auch entsprechende
    Sicherheitsrisiken darstellen.

    \item \textbf{Blueooth}
    Ebenso wie Wlan hat Bluetooth eine weite Verbreitung bei Endanwendergeräten. Durch Implementierung 
    des Standard Bluetooth LE ist es möglich leistungsschwache und Batteriebetriebene Geräte mit Bluetooth auszustatten. Bluetooth
    ist allerdings nicht für hohe Reichweite oder für viele Geräte konzipiert. Der geographische Raum, den eine Bluetooth
    Bridge abdecken kann, ist begrenzt durch die Reichweite des zentralen Gateways.

    \item \textbf{Z-Wave}
    
    Z-Wave ist funktional ähnlich wie Zigbee. Das Protokoll ist vollständig proprietäre. Der Hauptunterschied ist, dass Z-Wave in einem frei nutzbaren
    Low-Frequencie Band arbeitet und nicht wie ZigBee mit 2,4 Ghz Wlan interefiert. Die Reichweite ist größer als Zigbee, der Ma

    \item \textbf{ZigBee}
    Zigbee ist ein auf den 802.40.5 Standard aufbauendes Protokoll, welches grundlegend für die Anbindung vieler
    Geräte in einem großen räumlichen Areal konzipiert ist. Ein großer konzeptioneller Vorteil ist, dass bei 
    ZigBee ein Mesh-Netzwerk aufgebaut wird. Es können auch Geräte angebunden werden, die keine direkte Funkverbindung
    zum Koordinator haben. Zusätzlich sind Funktionen implementiert, welche das Management einer hohen Anzahl von Devices
    erleichtert.
    
    \item \textbf{Thread}
    Thread ist ein Newcomer. Es basiert ebenfalls auf den 802.15.4 Standard. Ebenso wie ZigBee ist es Meshfähig, ein
    entscheidentes Unterscheidungsmerkmal ist allerdings, dass die Geräte per IPv6 adressiert werden. Daher sind die Geräte
    theoretisch ohne die Verwendung einer Bridge aus einem herkömmlichen Ethernet Netzwerk erreichbar und addresierbar.
\end{itemize}

\section{Zigbee Anwendungen}

\subsection{Kommerzielle Anwendungen}

\subsubsection{Amazon Echo}
    Der Heimassistent Amazon Echo ab Generation 4 ist der einzige seiner Art, der eine Zigbee Integration hat und damit als Gateway und Koordinator dienen
    kann. Die Pendanten der Firmen Google, Microsoft und Apple benötigen ein dediziertes Zigbee Gateway.

\subsubsection{Phillips Hue}
    Phillips vertreibt unter dem Namen eine Zigbee Bridge und eine Vielzahl von Devices aus dem Segment Beleuchtung und Steckdosen. Die Hardware ist vergleichsweiße
    teuer.

\subsubsection{Dresden Electronic}
    Dresden Electronic bietet Software und Hardware zum Aufbau von Zigbee Netzwerken an. Es gibt Zigbee USB Adapter und RaspberryPi Hats mit ATMega Chips,
    sowie eine Steuerungssoftware \grqq deCONZ \grqq{}. Als komplette Produktlinie für den nicht technisch visierten Endkungenmarkt gibt es die Produktsparte
    \grqq Phoscon\grqq{}, hautpsächlich zur smarten Beleuchtung.

\subsubsection*{Weitere Hersteller}
Weitere bekannte Hersteller/Marken mit Zigbee Devices und Gateways:
\begin{itemize}
    \item \textbf{Logitech} - Harmony Hub
    \item \textbf{LIDL} - Silvercrest
    \item \textbf{TUYA} - Smart Life
    \item \textbf{Innr} - ZigBee Bridge
    \item \textbf{SONOFF} - Günstige Hardware jeder Art
    \item \textbf{homee} -  modular Smart Home Central
    \item \textbf{Osram} - Lightify
    \item \textbf{Ledvance} - Zigbeefähige Steckdosen und Lampen \grqq Smart+ \grqq{}
\end{itemize}

Nachteil dieser Lösungen ist, dass die Kompatiblität zu Geräten von Drittherstellern vollständig in der Hand des Herstellers ist. In der Regel ist aus
wirtschaftlichen Gründen die Unterstützung konkurrierender Hersteller nicht gewünscht. Es ist schwierig, bei Anschaffung eines dieser Systeme die Kompatiblität
anderer Geräte sicherzustellen, da offiziell meißt nur die Geräte aus dem eigenem Haus supported sind.

\subsection{Nicht kommerzielle Anwendungen}

Vorteil von OpenSource Anwendungen ist, dass diese durch eine Community gepflegt und Geräte von drittherstellern beliebig integriert werden können.
Grundlegend ist der Zigbee Standard universell, und die Kompatiblität von Geräten verschiedener Hersteller möglich.

\subsubsection{zigbee2Mqtt}

zigbee2Mqtt ist ein quelloffenes Projekt auf GitHub, welches aus einer Serveranwendung mit WebGUI, und einer Firmware für diverse Texas Instruments Chips besteht.
Grundlegende Koordinator Fähigkeiten sind auf der Hardware implementiert, Hardware Abstraktionen sowie die Weiterreichung von Nachrichten an ein MQTT Broker sind
in der Webanwendung implementiert. Auf der anderen Seite des MQTT Brokers, zur Visualisierung und Steuerung der Devices können Anwendungne wie Homeassistant oder ioHAB eingesetzt
werden. zigbee2Mqtt bietet über eine Entwicklerkonsole die Möglichkeit, Informationen zu sammeln und direkt Einflussnahme auf die Devices zu nehmen.

\subsubsection{ZHA}
ZHA ist ein direkt in HomeAssistant integriertes Plugin, um Zigbee Koordinatoren direkt in HomeAssistant einzubinden. Vorteil
von ZHA ist, dass ZigBee Chips mehrere Hersteller unterstützt werden. ZHA unterstützt neben Texas Instruments auch Hardware von Dresden Elektronik,
Silicon Labs, DIGI und ZiGate. ZHA ist für den Anwender extrem vereinfacht, es sind wenige technische Informationen ersichtlich oder konfigurierbar. 







