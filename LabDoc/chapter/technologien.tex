\chapter{Marktübersicht Technologien}

\grqq Internet of things\grqq{} beschreibt die Befähigung von Endgeräten mit Datennetzen zu kommunizieren. So können Waschmaschinen einen fertigen Waschgang kommunizieren,
oder ein Heizungsthermostat sich die Außentemperatur aus dem Wetterbericht beziehen. Viele Hersteller haben mittlerweile ein breites Portfolio an sogenannten \grqq Smart Devices \grqq{} und den dazu
je nach Übertragunsprotokoll notwendigen \grqq Bridge\grqq{} und einer entsprechenden App zur Steuerung. Hier ist der Hersteller Phillips mit seiner Produktmarke \grqq Hue\grqq{}
als Beispiel zu nennen. Die Produktgruppe umfasst eine Bridge mit zugehöriger App, sowie den klassischen Komponenten wie Lampen, Steckdose und Schalter.
Der Markt wurde durch Heimassistenten stark belebt. Amazons Alexa, der Google Echo Dot und die pendanten von Apple und Microsoft sind in immer mehr Haushalten zu finden.
Diese Heimassistenten können sich entweder mit den herstellerspezifischen Bridges verbinden, oder können direkt an PANs (Personal-Area-Networks) wie Zigbee
teilnehmen und die Geräte steuern. Dieses Kapitel soll einen Überblick über die relevanten Technologien am Markt verschaffen. 

\section{Funkprotokolle}
Aktuell gibt es mehrere Funkprotokolle, welche im Bereich IoT relevant sind. Dazu gehören:
\begin{itemize}
    \item \textbf{Wlan} \\
    Wlan ist ein verbreiteter und etablierter Standard, der überwiegend für die Anbindung mobiler Geräte an den
    Internetrouter dient. Dies macht es naheliegend, auch smarte Geräte per WLAN einzubinden. Wlan ist allerdings 
    optimiert für hohe Übertragungsraten und nicht für leistungsschwache Endgeräte. Dies ist
    insbesondere für batteriebetriene Geräte nachteilig. Bei WLAN ist es problematisch, viele Geräte mit geringer Bandbreite mit einen Access-Point zu verbinden.
    Die nutzbare Bandbreite für Geräte wie Notebooks sinkt damit ab. 

    \item \textbf{Blueooth}\\
    Ebenso wie Wlan hat Bluetooth eine weite Verbreitung. Durch Implementierung 
    des Standard Bluetooth LE ist es möglich leistungsschwache sowie batteriebetriebene Geräte mit Bluetooth auszustatten. Bluetooth
    ist allerdings nicht für hohe Reichweiten oder für Netzwerke mit vielen Teilnehmern konzipiert. Primäre Anwendungsfall ist zum Beispiel das Verbinden eines Headsets mit 
    einem Handy. 

    \item \textbf{Z-Wave}\\
    
    Z-Wave ähnelt technologisch Zigbee. Das Protokoll ist proprietär. Der Hauptunterschied ist, dass Z-Wave in einem frei nutzbaren
    Low-Frequency Band arbeitet und nicht wie ZigBee im 2,4 Ghz Band. Die Reichweite ist durch die geringere Trägerfrequenz höher.

    \item \textbf{ZigBee}\\
    Zigbee \cite{zigbee} ist ein auf den 802.40.5 Standard aufbauendes Protokoll, welches grundlegend für die Anbindung vieler leistungsschwacher
    Geräte in einem großen räumlichen Areal konzipiert ist. Ein großer konzeptioneller Vorteil ist, dass bei 
    ZigBee ein Mesh-Netzwerk aufgebaut wird. Es können auch Geräte angebunden werden, die keine direkte Funkverbindung
    zum Koordinator haben. Zusätzlich sind Funktionen implementiert, welche das Management einer hohen Anzahl von Devices
    erleichtert.
    
    \item \textbf{Thread}\\
    Thread ist ein Funkprotokoll welches ebenfalls auf den 802.15.4 Standard basiert. Ebenso wie ZigBee ist es Meshfähig, ein
    entscheidendes Unterscheidungsmerkmal ist allerdings, dass die Geräte per IPv6 adressiert werden. Daher sind die Geräte
    theoretisch ohne die Verwendung einer Bridge aus einem herkömmlichen Ethernet Netzwerk addresierbar.
\end{itemize}

\section{Zigbee Anwendungen}

\subsection{Kommerzielle Anwendungen}

\subsubsection{Amazon Echo}
    Der Heimassistent Amazon Echo ab Generation 4 ist der einzige seiner Art, der eine Zigbee Integration hat und damit als Gateway und Koordinator dienen
    kann. Die Pendanten der Firmen Google, Microsoft und Apple benötigen ein dediziertes Zigbee Gateway.

\subsubsection{Phillips Hue}
    Phillips vertreibt unter dem Namen eine Zigbee Bridge und eine Vielzahl von Devices aus dem Segment Beleuchtung und Steckdosen. 

\subsubsection{Dresden Electronic}
    Dresden Electronic bietet Software und Hardware zum Aufbau von Zigbee Netzwerken an. Es werden Zigbee USB Adapter und RaspberryPi Hats mit ATMega Chips angeboten,
    sowie eine Steuerungssoftware \grqq deCONZ \grqq{}. Als komplette Produktlinie für den Endanwender gibt es die Produktsparte
    \grqq Phoscon\grqq{}, hautpsächlich zur smarten Beleuchtung.

\subsubsection*{Weitere Hersteller}
Weitere bekannte Hersteller/Marken mit Zigbee Devices und Gateways:
\begin{itemize}
    \item \textbf{Logitech} - Harmony Hub
    \item \textbf{LIDL} - Silvercrest
    \item \textbf{TUYA} - Smart Life
    \item \textbf{Innr} - ZigBee Bridge
    \item \textbf{SONOFF} - Günstige Hardware jeder Art
    \item \textbf{homee} -  modular Smart Home Central
    \item \textbf{Osram} - Lightify
    \item \textbf{Ledvance} - Zigbeefähige Steckdosen und Lampen \grqq Smart+ \grqq{}
\end{itemize}

Nachteil dieser Lösungen ist, dass die Kompatiblität zu Geräten von Drittherstellern vollständig in der Hand des Herstellers ist. In der Regel ist aus
wirtschaftlichen Gründen die Unterstützung konkurrierender Hersteller nicht gewünscht. Es ist schwierig, bei Anschaffung eines dieser Systeme die Kompatiblität
anderer Geräte sicherzustellen, da offiziell meißt nur die Geräte aus dem eigenem Haus supported sind.

\subsection{Nicht kommerzielle Anwendungen}

Vorteil von quelloffenen Anwendungen ist, dass diese durch eine Community gepflegt und Geräte von drittherstellern beliebig integriert werden können.
Grundlegend ist der Zigbee Standard universell, und die Kompatiblität von Geräten verschiedener Hersteller möglich.

\subsubsection{zigbee2mqtt}

zigbee2mqtt \cite{z2m} ist ein quelloffenes Projekt auf GitHub. Die Anwendung kann anstelle von propietären Bridges als ZigBee-Gateway
eingesetzt werden. Die Anwendung kann ein ZigBee Netzwerk in der Rolle des Koordinators verwalten und die Gerätefunktionen per MQTT nach außen verfügbar machen.

\subsubsection{ZHA}
ZHA ist ein direkt in HomeAssistant integriertes Plugin, um Zigbee Koordinatoren direkt in HomeAssistant einzubinden. Vorteil
von ZHA ist, dass ZigBee Chips mehrerer Hersteller unterstützt werden. ZHA unterstützt neben Texas Instruments auch Hardware von Dresden Elektronik,
Silicon Labs, DIGI und ZiGate. ZHA ist für den Anwender komfortabler als zigbee2mqtt, es sind wenige technische Informationen ersichtlich oder konfigurierbar.
Die Endgeräte Kompatiblität ist derzeit schlechter als bei zigbee2mqtt.







